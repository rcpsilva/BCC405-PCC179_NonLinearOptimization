\documentclass{article}
\usepackage[utf8]{inputenc}
\usepackage[margin=1.2in]{geometry}
\usepackage{hyperref}

\usepackage{tikz}
\usetikzlibrary{positioning}

\usepackage{natbib}
\usepackage{graphicx}
\usepackage{amsmath}
\usepackage{listings}
\usepackage{xcolor}


\definecolor{codegreen}{rgb}{0,0.6,0}
\definecolor{codegray}{rgb}{0.5,0.5,0.5}
\definecolor{codepurple}{rgb}{0.58,0,0.82}
\definecolor{backcolour}{rgb}{0.95,0.95,0.92}
\definecolor{deepblue}{rgb}{0,0,0.5}
\definecolor{deepred}{rgb}{0.6,0,0}
\definecolor{deepgreen}{rgb}{0,0.5,0}

\lstdefinestyle{mystyle}{
    backgroundcolor=\color{white},   
    commentstyle=\color{codegreen},
    keywordstyle=\color{deepblue},
    numberstyle=\tiny\color{codegray},
    stringstyle=\color{deepgreen},
    emph={Agent,__init__,act,self,union,exists, scope},
    emphstyle=\color{deepred},
    basicstyle=\ttfamily\footnotesize,
    breakatwhitespace=false,         
    breaklines=true,                 
    captionpos=b,                    
    keepspaces=true,                 
    numbers=left,                    
    numbersep=5pt,                  
    showspaces=false,                
    showstringspaces=false,
    showtabs=false,                  
    tabsize=3
}

\lstset{style=mystyle}

\title{\vspace{-2 cm}Universidade Federal de Ouro Preto \\ Otimização Não-Linear \\ Prova 1}
\author{Prof. Rodrigo Silva}
\date{}


\begin{document}

\maketitle


\begin{enumerate}

\item Considere o problema de otimização a seguir:

\begin{equation}
     min\ f(x_1,x_2) = x_1^3 + x_2^3 + 2x_1^2 + 4x_2^2 + 6
\end{equation}

\begin{enumerate}
    \item Prove que este problem possui ótimo global.
    \item A função $f(\cdot)$ é uma função convexa? Demonstre.
    \item Encontre os pontos críticos de $f(\cdot)$.
    \item Para cada ponto crítico utilize as condições de segunda ordem para determinar se eles são pontos de mínimo, máximo ou de sela.
    \item É possível dizer qual o mínimo global desta função? Explique.
    \item Suponha que precisamos resolver este problema utilizando um método de maximização. Como podemos transformar este problema em um problema de maximização?
\end{enumerate}

\end{enumerate}

\end{document}

