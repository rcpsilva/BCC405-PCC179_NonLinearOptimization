\documentclass{article}
\usepackage[utf8]{inputenc}
\usepackage[margin=1.2in]{geometry}
\usepackage{hyperref}

\usepackage{listings}
\usepackage{xcolor}

\definecolor{codegreen}{rgb}{0,0.6,0}
\definecolor{codegray}{rgb}{0.5,0.5,0.5}
\definecolor{codepurple}{rgb}{0.58,0,0.82}
\definecolor{backcolour}{rgb}{0.95,0.95,0.92}

\lstdefinestyle{mystyle}{
    backgroundcolor=\color{backcolour},   
    commentstyle=\color{codegreen},
    keywordstyle=\color{magenta},
    numberstyle=\tiny\color{codegray},
    stringstyle=\color{codepurple},
    basicstyle=\ttfamily\footnotesize,
    breakatwhitespace=false,         
    breaklines=true,                 
    captionpos=b,                    
    keepspaces=true,                 
    numbers=left,                    
    numbersep=5pt,                  
    showspaces=false,                
    showstringspaces=false,
    showtabs=false,                  
    tabsize=2
}

\lstset{style=mystyle}


\usepackage{tikz}
\usetikzlibrary{positioning}

\usepackage{natbib}
\usepackage{graphicx}
\usepackage{amsmath}
\usepackage{amsmath}
\usepackage{amssymb}
\newcommand{\x}{\mathbf{x}}

\title{\vspace{-2 cm}Universidade Federal de Ouro Preto \\ PCC104 - Projeto e Análise de Algoritmos \\ Força Bruta e Busca Exaustiva}
\author{Prof. Rodrigo Silva}
%\date{}


\begin{document}

\maketitle

% \begin{itemize}
%     \item Implementar as atividades práticas em C++ é altamente recomendado.
%     \item Utilize ao máximo os algoritmos e estruturas de dados da biblioteca STL. \url{https://www.geeksforgeeks.org/the-c-standard-template-library-stl/}. 
%     \item Evite ao máximo a utilização de ponteiros, mas se precisar, utilizar ponteiros inteligentes \url{https://alandefreitas.github.io/moderncpp/basic-syntax/pointers/smart-pointers/}. 
%     \item Quando precisar de uma estrutura de dados linear sempre avalie primeiro a utilização da classe \texttt{vector} (\url{https://en.cppreference.com/w/cpp/container/vector})
% \end{itemize}


\section{Conceitos Básicos}

\begin{enumerate}
    \item Como resolver um problema de maximização como um problema de minimização?
    \item Qual a diferença entre problemas lineares e não-lineares?
    \item O que é função objetivo?
    \item O que é variável de projeto?
    \item O que é uma restrição?
    \item Em um gráfico de contorno, como definimos os contornos?
    \item Defina mínimo local.
    \item Defina máximo local.
    \item Defina o termo ``espaço de busca''.
    \item Defina mínimo local.
    \item Defina máximo local.
    \item Qual a diferença entre o mínimo e o minimizado de uma função?
    \item Qual a diferença entre um mínimo local e um mínimo global?
    \item  Sejam $g : \mathbb{R} \rightarrow \mathbb{R}$ uma função estritamente crescente e $f : \mathbb{R}^n \rightarrow \mathbb{R}$. Prove que minimizar $f(\x)$ é equivalente a minimizar $g(f(\x))$.
    \item Prove que minimizar $f(\x)$ é equivalente à minimizar $f(\x) + k$, onde $k$ é uma constante em $\mathbb{R}$.
    \item Para quais valores de $c$, minimizar $f(\x)$ é equivalente à minimizar $cf(\x)$? Explique.
    \item Considere a função \(f(x) = e^x - 2x\) para \(x \in \mathbb{R}\). Determine se existe um minimizador global de \(f(x)\) em \(\mathbb{R}\).
    %Para verificar se existe um minimizador global, primeiro vamos verificar se a função é coerciva.
    %Dado que \(\lim_{x \to \infty} e^x = \infty\), podemos afirmar que \(f(x) = e^x - 2x\) também tende ao infinito quando \(x\) tende ao infinito. Portanto, a função \(f(x)\) é coerciva.
    %Como a função é coerciva, de acordo com a propriedade mencionada, ela tem um minimizador global.
    \item Considere a função \(f(x) = \frac{1}{x}\) para \(x > 0\). Verifique se \(f(x)\) é coerciva.
    %Para verificar se a função é coerciva, precisamos determinar se \(\lim_{x \to \infty} f(x) = \infty\).
    %
    %\(\lim_{x \to \infty} \frac{1}{x} = 0\), o que significa que \(f(x)\) não tende ao infinito à medida que \(x\) tende ao infinito. Portanto, a função \(f(x) = \frac{1}{x}\) não é coerciva.
    \item  
\end{enumerate}



%\footnotetext{Livro - \textit{Introduction to the Design and Analysis of Algorithms (3rd Edition)}}

%\bibliographystyle{plain}
%\bibliography{references}
\end{document}

