\documentclass{article}
\usepackage[utf8]{inputenc}
\usepackage[margin=1.2in]{geometry}
\usepackage{hyperref}

\usepackage{listings}
\usepackage{xcolor}

\definecolor{codegreen}{rgb}{0,0.6,0}
\definecolor{codegray}{rgb}{0.5,0.5,0.5}
\definecolor{codepurple}{rgb}{0.58,0,0.82}
\definecolor{backcolour}{rgb}{0.95,0.95,0.92}

\lstdefinestyle{mystyle}{
    backgroundcolor=\color{backcolour},   
    commentstyle=\color{codegreen},
    keywordstyle=\color{magenta},
    numberstyle=\tiny\color{codegray},
    stringstyle=\color{codepurple},
    basicstyle=\ttfamily\footnotesize,
    breakatwhitespace=false,         
    breaklines=true,                 
    captionpos=b,                    
    keepspaces=true,                 
    numbers=left,                    
    numbersep=5pt,                  
    showspaces=false,                
    showstringspaces=false,
    showtabs=false,                  
    tabsize=2
}

\lstset{style=mystyle}


\usepackage{tikz}
\usetikzlibrary{positioning}

\usepackage{natbib}
\usepackage{graphicx}
\usepackage{amsmath}

\title{\vspace{-2 cm}Universidade Federal de Ouro Preto \\ PCC109 - Otimização não-linear \\ Taylor, Matrizes e Condições de Otimalidade}
\author{Prof. Rodrigo Silva}
%\date{}


\begin{document}

\maketitle

\section{Exercícios}

\begin{enumerate}
    \item Aproximações por série de Taylor
    \begin{enumerate}
        \item Considere a função multidimensional \(f(\mathbf{x}) = x_1^2 + 2x_2^2\) em torno do ponto \(\mathbf{x}_0 = (1, -1)\).
        \begin{enumerate}
            \item Calcule o valor da função \(f(\mathbf{x})\) no ponto \(\mathbf{x}_0\).
            \item Calcule o vetor gradiente \(\nabla f(\mathbf{x}_0)\).
            \item  Calcule a matriz Hessiana \(\mathbf{H}(\mathbf{x}_0)\).
            \item Use a expansão de série de Taylor de segunda ordem para aproximar \(f(\mathbf{x})\) em torno de \(\mathbf{x}_0\).
        \end{enumerate}

    \item Considere a função multidimensional \(f(\mathbf{x}) = x_1^3 + 2x_2^2 - x_1x_2\) em torno do ponto \(\mathbf{x}_0 = (2, 1)\).

        \begin{enumerate}
            \item Calcule o valor da função \(f(\mathbf{x})\) no ponto \(\mathbf{x}_0\).
            \item  Calcule o vetor gradiente \(\nabla f(\mathbf{x}_0)\).
            \item  Calcule a matriz Hessiana \(\mathbf{H}(\mathbf{x}_0)\).
            \item  Use a expansão de série de Taylor de segunda ordem para aproximar \(f(\mathbf{x})\) em torno de \(\mathbf{x}_0\).
            
        \end{enumerate}

    \item Considere a função multidimensional \(f(\mathbf{x}) = e^{x_1} + \sin(x_2)\) em torno do ponto \(\mathbf{x}_0 = (0, 0)\).
    
    \begin{enumerate}
        \item Calcule o valor da função \(f(\mathbf{x})\) no ponto \(\mathbf{x}_0\).
        \item Calcule o vetor gradiente \(\nabla f(\mathbf{x}_0)\).
        \item Calcule a matriz Hessiana \(\mathbf{H}(\mathbf{x}_0)\).
        \item Use a expansão de série de Taylor de segunda ordem para aproximar \(f(\mathbf{x})\) em torno de \(\mathbf{x}_0\).
    \end{enumerate}



    \end{enumerate}
    \item Definição de Matrizes
    \begin{enumerate}
        \item Considere a matriz abaixo:
        
                \[
                \begin{bmatrix}
                a_{11} & a_{12} & a_{13} \\
                a_{21} & a_{22} & a_{23} \\
                a_{31} & a_{32} & a_{33} \\
                \end{bmatrix}
                \]
            \begin{enumerate}
                \item Quais são os menores princiapais desta matriz?
                \item Quais são os menores principais líderes dessa matriz?
            \end{enumerate}
        \item Considere as matrizes abaixo:
        
        \[
                \begin{bmatrix}
                1 & 0 \\
                0 & 1 
                \end{bmatrix} , 
                \begin{bmatrix}
                -1 & 0 \\
                0 & -1 
                \end{bmatrix},  
                \begin{bmatrix}
                    1 & 0 \\
                    0 & -1
                \end{bmatrix} ,
                \begin{bmatrix}
                    1 & 0 \\
                    0 & 0
                \end{bmatrix} ,
                \begin{bmatrix}
                    -1 & 0 \\
                    0 & 0
                \end{bmatrix}
        \]

        Para cada uma delas:
        \begin{itemize}
            \item Escreva a forma quadrática e classique a matriz em $A>0$, $A<0$, $A\geq0$,$A\leq0$ ou $A$ indefinida.
            \item Calcule os menores principais líderes e observe o que acontece. 
        \end{itemize}
    \end{enumerate}
    \item Condições de Otimalidade
    \begin{enumerate}
        \item  Considere a função unidimensional \(f(x) = x^4 - 8x^3 + 18x^2\). 
        \begin{enumerate}
            \item Encontre todos os minimizadores locais da função e determine seus valores mínimos locais.
            \item Identifique se existe algum minimizador global e, se existir, encontre-o e determine o valor mínimo global.      
        \end{enumerate}
        \item Considere a função bidimensional \(f(x, y) = x^2 + y^2 - 4xy + 2x + 2y\).
        \begin{enumerate}
            \item Esta função possui minimizador? Justifique.
            \item Encontre todos os minimizadores locais da função e determine seus valores mínimos locais.
            \item Identifique se existe algum minimizador global e, se existir, encontre-o e determine o valor mínimo global.
        \end{enumerate}
        \item Considere a função bidimensional \(f(x, y) = (x^2 + y^2)^2 - 4x^2 - 4y^2\).
        \begin{enumerate}
            \item Esta função possui minimizador? Justifique.
            \item Encontre todos os minimizadores locais da função e determine seus valores mínimos locais.
            \item Identifique se existe algum minimizador global e, se existir, encontre-o e determine o valor mínimo global.
        \end{enumerate}
    \end{enumerate}
\end{enumerate}

\section{Notas}

Em forma vetorizada, a expansão de série de Taylor de segunda ordem para uma função multidimensional \(f(\mathbf{x})\) em torno do ponto \(\mathbf{x}_0\) pode ser expressa da seguinte forma:

\[
\begin{split}
f(\mathbf{x}) &\approx f(\mathbf{x}_0) + \nabla f(\mathbf{x}_0) \cdot (\mathbf{x} - \mathbf{x}_0) \\
&+ \frac{1}{2}(\mathbf{x} - \mathbf{x}_0)^T \mathbf{H}(\mathbf{x}_0)(\mathbf{x} - \mathbf{x}_0)
\end{split}
\]

Nessa expressão:

\begin{enumerate}
    \item \(f(\mathbf{x}_0)\) é o valor da função no ponto \(\mathbf{x}_0\).
    \item  \(\nabla f(\mathbf{x}_0)\) é o vetor gradiente de \(f\) avaliado no ponto \(\mathbf{x}_0\).
    \item  \(\mathbf{H}(\mathbf{x}_0)\) é a matriz Hessiana de \(f\) avaliada no ponto \(\mathbf{x}_0\).
    \item  \((\mathbf{x} - \mathbf{x}_0)\) representa o vetor de diferenças entre \(\mathbf{x}\) e \(\mathbf{x}_0\).
\end{enumerate}


A matriz Hessiana \(\mathbf{H}(\mathbf{x}_0)\) é uma matriz quadrada de segundas derivadas parciais e é definida como:

\[
\mathbf{H}(\mathbf{x}_0) = \begin{bmatrix}
\frac{\partial^2 f}{\partial x_1^2}(\mathbf{x}_0) & \frac{\partial^2 f}{\partial x_1 \partial x_2}(\mathbf{x}_0) & \ldots & \frac{\partial^2 f}{\partial x_1 \partial x_n}(\mathbf{x}_0) \\
\frac{\partial^2 f}{\partial x_2 \partial x_1}(\mathbf{x}_0) & \frac{\partial^2 f}{\partial x_2^2}(\mathbf{x}_0) & \ldots & \frac{\partial^2 f}{\partial x_2 \partial x_n}(\mathbf{x}_0) \\
\vdots & \vdots & \ddots & \vdots \\
\frac{\partial^2 f}{\partial x_n \partial x_1}(\mathbf{x}_0) & \frac{\partial^2 f}{\partial x_n \partial x_2}(\mathbf{x}_0) & \ldots & \frac{\partial^2 f}{\partial x_n^2}(\mathbf{x}_0)
\end{bmatrix}
\]

Essa forma vetorizada permite que você aproxime a função multidimensional \(f(\mathbf{x})\) como um polinômio quadrático nas proximidades do ponto \(\mathbf{x}_0\). O vetor gradiente e a matriz Hessiana capturam informações sobre as primeiras e segundas derivadas da função, tornando-a uma ferramenta poderosa para otimização e análise numérica em espaços multidimensionais.


%\footnotetext{Livro - \textit{Introduction to the Design and Analysis of Algorithms (3rd Edition)}}

%\bibliographystyle{plain}
%\bibliography{references}
\end{document}

