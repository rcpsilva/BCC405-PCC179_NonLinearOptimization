\documentclass{article}
\usepackage[utf8]{inputenc}
\usepackage[margin=1.2in]{geometry}
\usepackage{hyperref}

\usepackage{listings}
\usepackage{xcolor}

\definecolor{codegreen}{rgb}{0,0.6,0}
\definecolor{codegray}{rgb}{0.5,0.5,0.5}
\definecolor{codepurple}{rgb}{0.58,0,0.82}
\definecolor{backcolour}{rgb}{0.95,0.95,0.92}

\lstdefinestyle{mystyle}{
    backgroundcolor=\color{backcolour},   
    commentstyle=\color{codegreen},
    keywordstyle=\color{magenta},
    numberstyle=\tiny\color{codegray},
    stringstyle=\color{codepurple},
    basicstyle=\ttfamily\footnotesize,
    breakatwhitespace=false,         
    breaklines=true,                 
    captionpos=b,                    
    keepspaces=true,                 
    numbers=left,                    
    numbersep=5pt,                  
    showspaces=false,                
    showstringspaces=false,
    showtabs=false,                  
    tabsize=2
}

\lstset{style=mystyle}


\usepackage{tikz}
\usetikzlibrary{positioning}

\usepackage{natbib}
\usepackage{graphicx}
\usepackage{amsmath}

\title{\vspace{-2 cm}Universidade Federal de Ouro Preto \\ PCC179 - Otimização Não-Linear \\ Nivelamento}
\author{Prof. Rodrigo Silva}
%\date{}


\begin{document}

\maketitle

\section{Cálculo}

\begin{enumerate}
    \item Relembrando Funções \url{https://www.youtube.com/watch?v=vzS9JVgYytw&list=PLf1lowbdbFIALbQquDHawwdzPvUpobN5f}
    \item Limite \url{https://www.youtube.com/playlist?list=PLf1lowbdbFIB3iWi1lRWFHHQw249iAo6D}
    \item Derivadas \url{https://www.youtube.com/playlist?list=PLf1lowbdbFIAURvpD8Qy8PqwrMjwx0N64}
    \item Derivadas Parciais \url{https://www.youtube.com/watch?v=j9jjZHFasYE&list=PL67473CC34F0CC698}
\end{enumerate}

\section{Algebra linear}

\begin{enumerate}
    \item Vetores e Espaços \url{https://www.khanacademy.org/math/linear-algebra/vectors-and-spaces}
    \item Operações com Matrizes \url{https://www.khanacademy.org/math/linear-algebra/matrix-transformations}
    \item Projeções e Bases \url{https://www.khanacademy.org/math/linear-algebra/alternate-bases}
\end{enumerate}




%\bibliographystyle{plain}
%\bibliography{references}
\end{document}
